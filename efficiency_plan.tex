\documentclass{article}

\usepackage{amsmath}

\title{$D\rightarrow K3\pi$ Efficiency Reweighting: Amplitude Models}
\begin{document}
\maketitle
\begin{itemize}
    \item Motivation: we want to find the momentum-space efficiency function $\epsilon(\mathbf{x})$,
          where $\mathbf{x}$ is a point in the final state phase space.
    \item We have produced Monte Carlo (MC) samples for three categories of decay, right sign (RS) wrong sign (WS) and phase space (phsp).
    \item We know the PDFs\footnote[1]{None of these are really PDFs, since they aren't properly normalized.} that were used to generate the MC samples; denote these
          $\mathcal{A}_{RS}(\mathbf{x})$, $\mathcal{A}_{WS}(\mathbf{x})$ and $\mathcal{A}_{phsp}(\mathbf{x})$.
    \item The PDF\footnotemark\ that describes the MC samples is (by definition of the efficiency)\textsuperscript{i think?} the product of the generating PDF\footnotemark\ and the efficiency:
          \begin{equation*}
              p^{MC}_{RS}(\mathbf{x}) = \mathcal{A}_{RS}(\mathbf{x})\epsilon(\mathbf{x})
          \end{equation*}
          \begin{equation}
              p^{MC}_{WS}(\mathbf{x}) = \mathcal{A}_{WS}(\mathbf{x})\epsilon(\mathbf{x})
          \end{equation}
          \begin{equation*}
              p^{MC}_{phsp}(\mathbf{x}) = \mathcal{A}_{phsp}(\mathbf{x})\epsilon(\mathbf{x})
          \end{equation*}
    \item If we combine the MC samples, the PDF describing them is:
          \begin{equation}
              p^{MC}(\mathbf{x}) = (\mathcal{I}_{RS}\mathcal{A}_{RS}
              + \mathcal{I}_{WS}\mathcal{A}_{WS}
              + \mathcal{I}_{phsp}\mathcal{A}_{phsp})\epsilon(\mathbf{x})
          \end{equation}
          Where $\mathcal{I}_{i}$ are numbers that can be extracted from the MC and depend on the relative statistics.
    \item We can therefore extract the efficiency $\epsilon(\mathbf{x})$ by generating a sample (the "model sample") according to the PDF\footnotemark:
          \begin{equation}
              p^{model}(\mathbf{x}) = \mathcal{I}_{RS}\mathcal{A}_{RS}
              + \mathcal{I}_{WS}\mathcal{A}_{WS}
              + \mathcal{I}_{phsp}\mathcal{A}_{phsp}
          \end{equation}
          and reweighting the combined MC sample to the model sample.
    \item We have assumed here that the efficiency for the all samples are the same, but this was an implicit assumption anyway\footnote[2]{Otherwise we would be talking about efficiency model\textbf{s}, and the user of the efficiency fcn would have to specify whether they wanted RS/WS efficiency}.
\end{itemize}

\end{document}